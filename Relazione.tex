\documentclass[12pt]{article}
\usepackage{graphicx}

\usepackage{amsmath}
\usepackage{amsfonts}
\usepackage{graphicx}
\graphicspath{ {./imgs/plots/} }

\title{Relazione progetto}
\author{Apollonio Francesco\\Bianchi Andrea\\Mazzetti Francesca}

\begin{document}

\begin{titlepage}
\maketitle
\pagenumbering{gobble}
\end{titlepage}
\newpage
\tableofcontents
\newpage
\pagenumbering{arabic}

\section{Introduzione}

    Il problema di Deblur consiste nella ricostruzione di un'immagine a partire da un dato acquisito mediante il modello:
    \begin{align*}
        b = Ax + \eta
    \end{align*}
    Con $b$ che è l'immagine corrotta, $x$ l'immagine originale da ricostruire, la matrice $A$ applica il blur gaussiano e $\eta$ è il rumore aggiunto all'immagine sfocata, con distribuzione Gaussiana di media $\mathbb{O}$ e deviazione standard $\sigma$.
    
    \subsection{Il dataset}
    Per eseguire i test richiesti è stato generato un dataset di 8 immagini contenenti forme geometriche varie su sfondo nero a cui sono state aggiunte due ulteriori immagini fotografiche, delle quali una, la figura "sample9.png" molto contrastata e con pochi dettagli e l'altra, "sample10.png" meno contrastata ma più dettagliata.
    
    \subsection{Gli algoritmi}
    Per ricostruire le immagini danneggiate sono stati impiegati diversi algoritmi, in modo da poter poi confrontarne l'efficacia.
    
    \paragraph{La soluzione naive}
    Il primo tentativo di ricostruzione è stato fatto utilizzando un algoritmo semplice - per questo detto naive - per risolvere il problema di ottimizzazione:
    \begin{align*}
        x^* = \arg\min_x \frac{1}{2} ||Ax - b||_2^2
    \end{align*}
    In questo caso la funzione da minimizzare è $f(x) = \frac{1}{2} ||Ax - b||_2^2$ e il suo gradiente è $\nabla f(x) = A^TAx - A^Tb$. La funzione è stata implementata usando il metodo dei gradienti coniugati, implementato dalla funzione minimize inclusa nella libreria numpy.
    
    \paragraph{Regolarizzazione}
    Dal momento che la funzione naive recupera sì la nitidezza dell'immagine, ma introduce un rumore elevato, è necessario introdurre un termine di regolarizzazione di Tikhonov, il problema di minimizzazione diventa quindi:
    \begin{align*}
        x^* = \arg\min_x \frac{1}{2} ||Ax - b||_2^2 + \frac{\lambda}{2} ||x||_2^2
    \end{align*}
    La funzione da minimizzare è quindi $f(x) = \frac{1}{2} ||Ax - b||_2^2 + \frac{\lambda}{2} ||x||_2^2$ e il suo gradiente è $\nabla f(x) = A^TAx - A^Tb + \lambda x$.
    La funzione è stata implementata sia tramite la funzione minimize di numpy che tramite il metodo del gradiente implementato a lezione. Sono poi stati eseguiti test con diversi valori di lambda.
    
    \paragraph{Variazione totale}
    Un altro termine di regolarizzazione adatto è dato dalla funzione di Variazione Totale. Data $x$ l'immagine di dimensioni $n\times m$, la variazione totale $TV$ di $x$è definita come:
    \begin{align*}
        TV(u) = \sum_i^n{\sum_j^m{\sqrt{||\nabla u(i, j)||_2^2 + \epsilon^2}}}
    \end{align*}
    Il problema di minimo da risolvere diventa quindi:
    \begin{align*}
        x^* = \arg\min_x \frac{1}{2} ||Ax - b||_2^2 + \lambda TV(u)
    \end{align*}
    il cui gradiente è:
    \begin{align*}
    \nabla f(x) = (A^TAx - A^Tb)  + \lambda \nabla TV(x)
    \end{align*}
    La funzione è stata implementata usando il metodo del gradiente implementato a lezione e già usato per il punto precedente. Sono stati eseguiti test per diversi valori di $\lambda$.
    Infine, per risolvere il problema è stato necessario anche calcolare il gradiente della variazione totale, che è dato da:
    \begin{align*}
        x^* = \arg\min_x \frac{1}{2} ||Ax - b||_2^2 + \lambda TV(u)
    \end{align*}
    il cui gradiente $\nabla f$ è dato da
    \begin{align*}
        \nabla f(x) = (A^TAx - A^Tb)  + \lambda \nabla TV(x)
    \end{align*}
    
    \subsection{I test}
    Dopo aver applicato il blur gaussiano e il disturbo alle immagini del dataset, abbiamo applicato diversi algoritmi per migliorare la qualità delle immagini. Per ogni immagine abbiamo eseguito un ciclo di test applicando diversi blur gaussiani, diversi valori di deviazione standard per il rumore e diversi valori per il parametro $\lambda$ del termine di regolarizzazione di Tikhonov. 
    I valori usati sono riassunti nella seguente tabella:
    
    \begin{center}
    \begin{tabular}{||c c c c||} 
     \hline
     Dim Kernel & Std dev Sigma Kernel & Std Dev Rumore & Lambda \\ [0.5ex] 
     \hline\hline
     5 $\times$ 5 & 0.5 & 0.01 & 0.01 \\ 
     \hline
     7 $\times$ 7 & 1 & 0.02 & 0.05 \\
     \hline
     9 $\times$ 9 & 1.3 & 0.03 & 0.08 \\
     \hline
     N.A. & N.A. & 0.04 & 0.32 \\
     \hline
     N.A. & N.A. & 0.05 & 1 \\ [0.2ex] 
     \hline
    \end{tabular}
    \end{center}
    
    I dati raccolti possono essere trovati integralmente nella cartella "data" allegata, di seguito commenteremo i risultati più rilevanti.

\section{Analisi dei risultati}
    

\end{document}